\documentclass[Afour,sageh,times]{sagej}
\usepackage[utf8]{inputenc}
\usepackage[USenglish]{babel}
\usepackage{booktabs}
\usepackage{setspace}
\usepackage{csquotes}
\usepackage{url}
\usepackage{moreverb}
\usepackage[colorlinks,bookmarksopen,bookmarksnumbered,citecolor=red,urlcolor=red]{hyperref}
\newcommand\BibTeX{{\rmfamily B\kern-.05em \textsc{i\kern-.025em b}\kern-.08em T\kern-.1667em\lower.7ex\hbox{E}\kern-.125emX}}


\def\volumeyear{\the\year}




\begin{document}

\runninghead{Author Redacted}
\title{Nonplanarity and Street Network Representation in Urban Form Studies}
\author{Author Redacted \affilnum{1}}
\affiliation{\affilnum{1}Affiliation redacted}
\corrauth{Author Redacted, Address Redacted}
\email{Email Redacted}


\begin{abstract}

Mathematical models of street networks, called graphs, are widely used throughout the urban studies literature to examine travel behavior, accessibility, urban design patterns, and morphology. These graphs are commonly described as \enquote{planar,} meaning that they can be represented in two dimensions without any underpasses or overpasses. However, real urban street networks exist in three-dimensional space and frequently include grade separation such as bridges and tunnels. How well do planar graphs model real-world street networks? What does the extent of nonplanarity tell us about a city's infrastructure and urban development? We measure the nonplanarity of drivable and walkable street networks in the centers of 50 cities worldwide, then examine the variation of planarity across a single city. We find that while some street networks are approximately planar, many are poorly-modeled by planar graphs. In particular, planarity can inconsistently but drastically misrepresent routing, connectivity, intersection counts/densities, and block lengths. We develop two measures of a network's nonplanarity and argue it can be a useful indicator of transportation infrastructure and development.

\end{abstract}

\keywords{street network, GIS, urban form, transportation, urban design}

\maketitle

\section{Introduction}

In urban planning and design research, street networks are routinely used to calculate accessibility between origins and destinations or to compute indicators of the urban form, such as block sizes or intersection density and connectivity. Mathematical models of street networks, called graphs, have grown ubiquitous in the urban studies literature in recent years as they have been used to model household travel patterns, access and equity, pedestrian volume, urban design patterns and spatial morphology, and location centrality and polycentricity \citep{marshall_street_2010,porta_alterations_2014,marshall_community_2014,hajrasouliha_impact_2015,parthasarathi_street_2015,knight_metrics_2015,gil_street_2016,zhong_revealing_2017}.

In the urban studies literature, street networks are typically referred to as \enquote{planar} or \enquote{approximately planar} (see Table \ref{tab:planar_quotes}), meaning that they can be well-modeled by a flat two-dimensional model that inherently precludes overpasses or underpasses. Of course in the real world, urban street networks are embedded in three-dimensional space and often feature grade separation, bridges, and tunnels. This leads to two intertwined questions. First, how well do these planar graphs model urban street networks? Second, what does the extent to which a network is nonplanar tell us about a city's infrastructure and development?

This study tests the planarity of street networks around the world. It also presents two new measures of the degree of nonplanarity that can be generalized to other types of spatial networks. These indicators help describe the nature of the urban form and transportation infrastructure. Despite common claims in the literature that street networks are planar graphs, this study finds that they generally are nonplanar and that planar graphs poorly model the street networks of many cities. Further, the magnitude of this bias varies substantially across cities and urbanization types.

This article is organized as follows. The following section introduces the basics of graph theory relevant to urban studies, focusing on discussions in the research literature about street network planarity. The next section discusses the methods used to acquire and analyze the street networks in this study. Then we present the results of this analysis before concluding with a discussion of their ramifications for street network research and urban form studies.


\section{Background and Motivation}

Graph theory is the mathematical study of networks \citep{newman_networks:_2010}. Graphs can model real-world networks such as friendships, the world wide web, or spatial networks such as urban street networks \citep{barthelemy_spatial_2011}. A graph $G$ consists of a set of nodes $N$ connected to one another by a set of edges $E$. An edge $uv$ in a directed graph points in one direction from some node $u$ to some node $v$, but an undirected graph's edges all point mutually in both directions. In a street network, the nodes represent intersections and dead-ends, and the (directed) edges represent the street segments that connect them. How a graph's nodes and edges connect to one another defines its \emph{topology}. For example, a node's \emph{degree} is a topological trait that represents how many edges connect to that node. A \emph{planar} graph can be drawn on a two-dimensional plane without any of its edges crossing each other, except where they intersect at nodes. If the graph cannot be drawn --- or redrawn --- to meet this criterion, then the graph is nonplanar \citep{trudeau_introduction_1994}. Street networks are embedded in space, which provides them with geometry --- such as geographical coordinates, lengths, areas, and angles --- along with their topology.

\begin{table*}[htbp]
\centering
\caption{Recent statements in the research literature regarding the representation of street networks as planar graphs.}
\label{tab:planar_quotes}
\input{table_planar_quotes.tex}
\end{table*}

This creates a minor wrinkle when we consider planarity: we must distinguish between a graph's mathematical/topological planarity, which we refer to as \emph{formal planarity}, and the planarity of its particular spatial embedding, which we refer to as \emph{spatial planarity}. For example, a street network might be spatially nonplanar due to its embedding in space (i.e., it contains overpasses or underpasses in the real world), but it could still be formally planar. If we \enquote{redraw} the graph by moving its nodes and edges around in space without changing how they are connected to one another (i.e., altering its geometry without altering its topology), there may exist some spatial embedding that prevents edges crossing anywhere but at nodes \citep[p.~6]{barthelemy_morphogenesis_2017}. In such a case, the street network is formally planar from a topological perspective, but its real-world embedding is spatially nonplanar.

\begin{figure}[tbp]
	\center
	\includegraphics[width=0.48\textwidth]{planar_vs_not.png}
	\caption{Formally and spatially planar (left) and nonplanar (right) road networks.}
	\label{fig:planar_vs_not}
\end{figure}

Consider the examples in Figure \ref{fig:planar_vs_not}. The left network comprises a set of paths and streets without any bridges or tunnels. In two-dimensions, its edges intersect each other only at nodes so it is by definition planar. The right network comprises surface streets and a grade-separated freeway interchange with many overpasses and underpasses. In two-dimensions, its edges sometimes cross each other at non-nodes (i.e., overpasses), so it is spatially nonplanar. Furthermore it is impossible to redraw the graph by moving its nodes and edges around in space such that its edges only intersect at nodes. Therefore it is also formally nonplanar. However, at a city- or region-wide spatial scale, these nonplanar edge crossings may be relatively uncommon; we might call such a network \emph{approximately} planar. Approximate planarity constrains nonplanar spatial networks such that they do not exhibit certain characteristics found among nonplanar aspatial graphs, such as small-world effects or power-law distributed node degrees \citep{crucitti_centrality_2006,fischer_spatial_2014}.

In the urban studies literature, street networks are commonly referred to as planar graphs. Table \ref{tab:planar_quotes} presents a survey of statements and reasoning around this claim. Some authors prefer to hedge slightly, arguing that street networks are \emph{approximately} or \emph{essentially} planar graphs that are close enough to be well-modeled as such.

If street networks can be sufficiently well-modeled by planar graphs, there are certain methodological benefits to doing so. Planar graphs offer computational simplicity and tractability. They enable easy polygonal spatial analysis of city blocks and form \citep{fohl_non-planar_1996,barthelemy_paths_2017} as well as the Euler characteristic. In mathematics, there is a bijection between planar graphs and trees, and classifying planar graphs presents a trivial problem \citep{louf_typology_2014}. Planar graphs are easier to visualize and can be faster to run algorithms on \citep{liebers_planarizing_2001}. Accordingly, \citet[p.~3]{barthelemy_spatial_2011} argues that \enquote{planar spatial networks are the most important and most studies have focused on these examples}. But in contrast, \citet{masucci_random_2009} and \citet{masucci_limited_2013} argue that planar graphs remain a compelling research domain for urban scholars because they were understudied until recently for two reasons: they appear topologically trivial and planarity does not lend itself to certain popular graph-theoretic analyses. Discussing the open research area around street networks as planar graphs, \citet[p.~1]{viana_simplicity_2013} state, \enquote{there is still a lack of global, high-level metrics allowing to characterize their structure and geometrical patterns.}

Despite the computational and mathematical advantages of simple planar models, street networks are often nonplanar in reality: many include at least one overpass or underpass that results in the failure of formal proofs of their planarity, such as the \citet{kuratowski_sur_1930} theorem or the \cite{hopcroft_efficient_1974} algorithm \citep[cf.][]{gastner_spatial_2006}. As \citet[p.~7]{levinson_network_2012} points out, \enquote{Real networks are neither perfect, nor planar, nor grids, though they may approximate them.}

Other authors have commented on this characteristic of street networks. \citet[p.~199]{jiang_object-oriented_2010} note that \enquote{quite often the transportation network has overpasses and underpasses that require a non-planar network representation.} \citet[p.~1258]{fischer_spatial_2014} explains that \enquote{for many infrastructure networks, {[planarity]} is approximately true, although bridges and tunnels in ground-transport networks are an obvious (but generally minor) exception.} However, the presence of such nonplanar elements can vex models. \enquote{The planar network data model has received widespread acceptance and use. Despite its popularity, the model has limitations for some areas of transportation analysis, especially where complex network structures are involved. One major problem is caused by the planar embedding requirement... intersections at grade cannot be distinguished from intersections with an overpass or underpass that do not cross at grade} \citep[p.~395]{fischer_gis_2004}. Twenty years ago, \citet[p.~18]{fohl_non-planar_1996} called for a nonplanar model to better represent truly nonplanar spatial networks. 

If a planar graph models a street network poorly, it could do so in multiple ways. Forcing planarity on a nonplanar street network creates artificial nodes in the graph at bridges and tunnels, which breaks routing. As \citet[p.~6]{kwan_review_1996} describe it, \enquote{the difficulty in accurately representing overpasses or underpasses may lead to problems when running various routing algorithms (e.g. recommending that a traveler make a left-turn at an intersection that proves to be an overpass)}. Intersection counts and densities will be overestimated in the presence of these false nodes. Consequently, edge lengths will be underestimated due to these artificial breakpoints splitting up street segments. Finally, this bias would likely behave inconsistently across different kinds of cities and street network types based on the extents to which they are planar.

Given these issues, what do \emph{approximately planar} and \emph{well-modeled} mean for street network research? How close is \enquote{close enough} for a planar graph to competently model a formally nonplanar street network? Do the biases of planar models behave consistently across geographies and development eras or do they misrepresent different cities to different degrees? And if street networks are at least generally approximately planar, how can we measure just how planar or nonplanar a given street network is?

The graph theory literature offers some measures of how \enquote{far off} a nonplanar graph $G$ is from being planar, including its \emph{crossing number} --- the minimum number of edge crossings of any drawing of $G$ --- and its \emph{skewness} --- the minimum number of edges that must be removed from $G$ to produce a planar graph \citep{liebers_planarizing_2001,chimani_non-planar_2009}. However, these measures are imperfect and hard to compute \citep{szekely_successful_2004,chimani_vertex_2012}. They also fail to correct for the size, density, or real-world embedding of a spatial network. In his discussion of road networks and approximate spatial planarity, \citet[p.~133]{newman_networks:_2010} argues that due to these drawbacks \enquote{no widely accepted metric for degree of planarity has emerged,} and calls for the development of better indicators.

Such measures would be particularly useful for street networks, as the extent to which a network is (or is not) planar can characterize the nature of its circulation infrastructure and urban form. For instance, late 20th-century freeway-oriented American cities might exhibit lower planarity than older walkable European cities. What about informal settlements in developing countries or rapidly urbanizing Chinese cities? Beyond the question of graph model goodness-of-fit, such indicators could provide useful information about urban development, civil infrastructure, and transportation system character.



\section{Methods}

In this study, we develop two new measures of the extent to which a spatial network is planar. We then analyze various world cities to better understand how well planar graphs model their street networks as well as the extent to which bias (i.e., model misrepresentation) varies across different places and types of urbanization. Finally, we consider what these indicators suggest about the nature of the urban form and transportation infrastructure in different cities.

\subsection{Data}

Following \citet{jacobs_great_1995} and \citet{cardillo_structural_2006}, we analyze a consistently sized, square-mile network at the centers of 50 cities worldwide. This allows us to consistently examine central urban street networks without being swamped by metropolitan-scale variation or the idiosyncrasies of individual municipalities' spatial extents. The 50 sampled cities span Africa, Asia, Australia, Europe, and North and South America. We look separately at each city's drivable and walkable street networks. For cities such as Moscow, with a newer commercial central business district (CBD) that lies apart from an \enquote{old town} center, we take the modern CBD as the city center.

To acquire these street networks, we use OSMnx to download the data for each city and network type from OpenStreetMap. OpenStreetMap is a collaborative online mapping platform commonly used by researchers because of its good worldwide coverage \citep{haklay_how_2010,jokar_arsanjani_openstreetmap_2015}. OSMnx is a Python-based software tool that allows us to automatically download a street network from OpenStreetMap for any study site in the world, and automatically process it into a length-weighted nonplanar directed graph \citep{boeing_osmnx:_2017}. It differentiates between walkable and drivable routes in the circulation network based on individual elements' metadata that describe how the route may be used. Thus the walkable network may contain surface streets, paths through parks, pedestrian passageways between buildings or under roads, and other walkable paths. The drivable network may contain surface streets, grade-separated freeways, and other drivable routes.

OpenStreetMap's raw data contain many interstitial nodes in the middle of street segments (forming an expansion graph) to allow streets to curve through space via a series of straight-line approximations. OSMnx automatically simplifies each graph's topology to retain nodes only at intersections and dead-ends, while faithfully retaining each edge's true spatial geometry. This provides an accurate count of intersections and an accurate measure of edge lengths for comparison between the planar and nonplanar representations of our networks.

\subsection{Analysis}

Once we have acquired and prepared our networks, we calculate three measures of planarity. The first is an aspatial test of formal planarity using the algorithm described by \citet{boyer_subgraph_2012}. This assesses if it is possible to rearrange the graph's nodes and edges in space, while preserving its topology, so that edges cross only at nodes. This binary true/false indicator tells us if the graph is formally planar, ignoring its real-world spatial embedding. However, street networks \emph{are} spatially embedded. Accordingly, for this study, we have developed a second and third indicator to assess the \enquote{extent} to which they are planar.

The second measure is the spatial planarity ratio $\phi$. It represents the ratio of nonplanar intersections $i_n$ (i.e., non-dead-end nodes in the nonplanar, three-dimensional, spatially-embedded graph) to planar intersections $i_p$ (i.e., edge crossings in the planar, two-dimensional, spatially-embedded graph): 

\begin{equation}
	\label{eq:spr}
	\phi = \frac{i_n}{i_p}
\end{equation}

Thus, $\phi$ represents what proportion of the two-dimensional edge crossings in the planar graph are true intersections in the nonplanar graph. This indicates the extent to which planarity overstates intersections and connectivity in a street network. A truly planar network with no bridges or tunnels will have an $\phi$ score of 1.0, while lower values indicate the percentage extent to which the network is planar. From this indicator, we can further calculate by what percentage the planar graph overstates intersection counts as:

\begin{equation}
	\label{eq:spr_overstates}
	1 - \frac{1}{\phi}
\end{equation}

The third measure is the edge length ratio ($\lambda$). It represents the ratio of the mean edge length in the planar graph $l_p$ to the mean edge length in the nonplanar graph $l_n$:

\begin{equation}
	\label{eq:elr}
	\lambda = \frac{l_p}{l_n}
\end{equation}

This indicates the extent to which planarity understates edge lengths in each street network by fragmenting street segments at overpasses or underpasses. A truly planar street network with no bridges or tunnels will thus have an $\lambda$ score of 1.0, while lower values indicate the percentage extent to which the network is planar. From the $\lambda$, we can further calculate by what percentage the planar graph understates the average edge length as:

\begin{equation}
	\label{eq:elr_understates}
	1 - \lambda
\end{equation}

Finally, we explore how these three indicators vary across a single city. To do so we analyze the drivable street network of Oakland, California as a case study. Oakland is a reasonably representative midsized American city with a variety of urban form types from gridded street patterns in its flatlands, to winding culs-de-sac in its hills, to freeways and dense blocks around its downtown. First we analyze the entire city of Oakland. Then we recreate the aforementioned methodology by randomly sampling 100 points within the city limits and analyzing the square-mile street networks centered on each. The resulting statistical dispersion of planarity demonstrates the extent to which analyzing an entire city's neighborhoods as a single graph may obscure neighborhood-scale infrastructure characteristics.



\section{Results}

\begin{table*}[htbp]
	\centering
	\caption{Planarity measures for the central street networks of 50 cities worldwide (Planar = whether street network passed the formal test of planarity; $\phi$ = spatial planarity ratio; $\lambda$ = edge length ratio).}
	\label{tab:world_cities}
	\begin{tabular}{ l l r r r r r r  }
\toprule
         &               & \multicolumn{3}{|c|}{Drive}         & \multicolumn{3}{c}{Walk}            \\
\midrule
Country      & City          &  Planar  &  $\phi$   &  $\lambda$   &  Planar  &  $\phi$   &  $\lambda$   \\
	\midrule
	Argentina & Buenos Aires &      Yes &  1.000 &  1.000 &       No &  0.939 &  0.941 \\
	Australia & Sydney &       No &  0.729 &  0.735 &       No &  0.902 &  0.884 \\
	Brazil & Sao Paulo &       No &  0.771 &  0.772 &       No &  0.824 &  0.803 \\
	Canada & Toronto &      Yes &  0.922 &  0.946 &       No &  0.838 &  0.824 \\
	& Vancouver &       No &  0.930 &  0.948 &       No &  0.923 &  0.920 \\
	Chile & Santiago &       No &  0.873 &  0.885 &       No &  0.967 &  0.965 \\
	China & Beijing &       No &  0.818 &  0.846 &       No &  0.842 &  0.842 \\
	& Hong Kong &       No &  0.838 &  0.823 &       No &  0.823 &  0.794 \\
	& Shanghai &       No &  0.682 &  0.708 &       No &  0.660 &  0.641 \\
	Denmark & Copenhagen &      Yes &  0.992 &  0.988 &       No &  0.994 &  0.985 \\
	Egypt & Cairo &       No &  0.897 &  0.913 &       No &  0.899 &  0.886 \\
	France & Lyon &       No &  0.995 &  0.991 &       No &  0.958 &  0.953 \\
	& Paris &       No &  0.982 &  0.987 &       No &  0.928 &  0.907 \\
	Germany & Berlin &       No &  0.939 &  0.945 &       No &  0.939 &  0.931 \\
	India & Delhi &      Yes &  1.000 &  1.000 &      Yes &  0.997 &  0.989 \\
	Indonesia & Jakarta &      Yes &  0.990 &  0.994 &       No &  0.969 &  0.965 \\
	Iran & Tehran &       No &  0.962 &  0.973 &       No &  0.953 &  0.951 \\
	Italy & Bologna &      Yes &  1.000 &  1.000 &      Yes &  0.996 &  0.996 \\
	& Florence &      Yes &  0.993 &  0.994 &       No &  0.980 &  0.974 \\
	& Milan &      Yes &  1.000 &  1.000 &       No &  0.849 &  0.832 \\
	Japan & Osaka &       No &  0.868 &  0.868 &       No &  0.953 &  0.950 \\
	& Tokyo &       No &  0.925 &  0.919 &       No &  0.924 &  0.912 \\
	Kenya & Nairobi &       No &  0.974 &  0.971 &       No &  0.949 &  0.938 \\
	Mexico & Mexico City &       No &  0.940 &  0.949 &       No &  0.912 &  0.917 \\
	Nigeria & Lagos &       No &  0.960 &  0.972 &       No &  0.990 &  0.988 \\
	Peru & Lima &       No &  0.941 &  0.951 &       No &  0.923 &  0.921 \\
	Philippines & Manila &       No &  0.940 &  0.947 &       No &  0.897 &  0.883 \\
	Russia & Moscow &       No &  0.540 &  0.596 &       No &  0.842 &  0.833 \\
	Singapore & Singapore &       No &  0.864 &  0.869 &       No &  0.891 &  0.880 \\
	Somalia & Mogadishu &      Yes &  1.000 &  1.000 &      Yes &  1.000 &  1.000 \\
	South Africa & Johannesburg &       No &  0.847 &  0.877 &       No &  0.997 &  0.997 \\
	Spain & Barcelona &      Yes &  1.000 &  1.000 &       No &  0.925 &  0.917 \\
	Switzerland & Geneva &       No &  0.985 &  0.981 &       No &  0.834 &  0.807 \\
	Thailand & Bangkok &       No &  0.993 &  0.979 &       No &  0.990 &  0.984 \\
	Turkey & Istanbul &       No &  0.965 &  0.964 &       No &  0.973 &  0.964 \\
	UAE & Dubai &       No &  0.679 &  0.668 &       No &  0.852 &  0.837 \\
	UK & Edinburgh &       No &  0.974 &  0.965 &       No &  0.987 &  0.983 \\
	& London &       No &  0.976 &  0.980 &       No &  0.853 &  0.836 \\
	USA & Atlanta &       No &  0.720 &  0.765 &       No &  0.805 &  0.788 \\
	& Chicago &       No &  0.748 &  0.786 &       No &  0.792 &  0.787 \\
	& Cincinnati &       No &  0.723 &  0.746 &       No &  0.929 &  0.922 \\
	& Dallas &       No &  0.584 &  0.639 &      Yes &  0.961 &  0.956 \\
	& Los Angeles &       No &  0.581 &  0.627 &       No &  0.784 &  0.785 \\
	& Miami &       No &  0.641 &  0.657 &       No &  0.962 &  0.961 \\
	& New York &       No &  0.879 &  0.878 &       No &  0.928 &  0.923 \\
	& Phoenix &       No &  0.949 &  0.958 &       No &  0.977 &  0.972 \\
	& San Francisco &       No &  0.935 &  0.937 &       No &  0.942 &  0.935 \\
	& Seattle &       No &  0.728 &  0.771 &       No &  0.931 &  0.924 \\
	& Washington DC &       No &  0.948 &  0.956 &       No &  0.964 &  0.961 \\
	Venezuela & Caracas &       No &  0.953 &  0.957 &      Yes &  1.000 &  1.000 \\
	\bottomrule
\end{tabular}
\end{table*}

\begin{figure*}[htbp]
	\center
	\includegraphics[width=\textwidth]{world_map_phi_bw.png}
	\caption{Map of world cities from Table \ref{tab:world_cities} grouped by $\phi$ terciles (lower values mean less planar).}
	\label{fig:world_map_bw}
\end{figure*}

\begin{figure}[htbp]
	\includegraphics[width=0.48\textwidth]{regression_phi_split.png}
	\caption{Log-log plot of $\lambda$ vs $\phi$, by network type, with simple regression lines.}
	\label{fig:regression_split}
\end{figure}

Table \ref{tab:world_cities} lists the $\phi$, $\lambda$, and formal planarity of the walkable and drivable networks in these 50 city centers. Among the drivable street networks, only 20\% are formally planar. On average, they are 88.5\% planar by the $\phi$ measure and 90\% planar by the $\lambda$ measure. The individual $\phi$ values indicate that spatial planarity ranges from a high of 100\% in seven of these cities to a low of 57\% in Moscow. The $\lambda$ values indicate that spatial planarity ranges from a high of 100\% in seven of these cities to a low of 63.5\% in Los Angeles. On average across these networks, planar representations overcount intersections by 16\% and underestimate street segment lengths by 10\% (Equations \ref{eq:spr_overstates} and \ref{eq:elr_understates}).

Among walkable street networks, only 10\% are formally planar. On average, they are 92\% planar by the $\phi$ measure and 91.5\% planar by the $\lambda$ measure. The individual $\phi$ values indicate that spatial planarity ranges from a high of 100\% in two cities to a low of 67\% in Shanghai. The $\lambda$ values indicate that spatial planarity ranges from a high of 100\% in two cities to a low of 66\% in Shanghai. On average across these networks, planar representations overcount intersections by 9\% and underestimate street segment lengths by 8.5\%. Fewer walkable than drivable networks are formally planar, but on average these walking networks are slightly more spatially planar than the driving networks.

Not all formally planar street networks are spatially planar. For example, Toronto's drivable network is formally planar but only 93\% ($\phi$) and 96\% ($\lambda$) spatially planar. In total, three drivable networks (Toronto, Jakarta, and Copenhagen) and three walkable networks (Dallas, Delhi, and Bologna) are formally planar but spatially nonplanar to various extents. About a third of the cities studied demonstrate $\phi$ spatial planarity of 97\% or higher, suggesting that they are \enquote{approximately} planar. However, another third of the cities studied are less than 87\% planar. Dallas, Los Angeles, and Moscow have $\phi$ values below 60\%, suggesting planar graphs poorly model these city centers. Moreover, planar graphs overstate the intersection counts in these three networks by 67\%, 72\%, and 74\% respectively.

Mogadishu is the only city studied that demonstrates perfect planarity across all three indicators for both network types. All three Italian cities demonstrate perfect planarity in their centers' drivable networks, but not in their walkable networks. The extent of planarity is not consistent across network types: Dallas's walkable $\phi$ is 61\% greater than its drivable $\phi$, while Geneva's drivable $\phi$ is 19\% greater than its walkable $\phi$. Figure \ref{fig:world_map_bw} maps the distribution of $\phi$ values around the world. While nearly every European city is in the highest tercile, indicating their networks are more planar, most American cities are in the lowest tercile, indicating their networks are more nonplanar.

Figure \ref{fig:regression_split} depicts the relationship between the $\phi$ and $\lambda$ indicators across all 50 cities for both network types, as a log-log plot. The indicators' log-log relationship is linear, positive, and very strong (drivable $r^2=0.98$ and walkable $r^2=0.99$). The coefficients of determination tell us that these two indicators unsurprisingly provide redundant statistical information about the extent to which a network is planar. However, each assesses different implications of this bias for measuring the urban form.

\begin{table}[htbp]
	\centering
	\caption{Summary statistics of planarity indicators across 100 random samples of Oakland, California's drivable network.}
	\label{tab:samples_city}
	\begin{tabular}{lrr}
\toprule
         & $\phi$ &  $\lambda$   \\
\midrule
count    &  100   &  100 \\
mean     &  0.929 &  0.939 \\
$\sigma$ &  0.103 &  0.089 \\
min      &  0.569 &  0.567 \\
max      &  1.000 &  1.000 \\
\bottomrule
\end{tabular}
\end{table}

Finally, we examine how these measures behave across an entire city. We find that Oakland's city-wide street network is formally nonplanar. The city has a $\phi$ score of 91.8\% and an $\lambda$ score of 93.6\%. This suggests that the planar representation of Oakland's drivable street network overstates the number of intersections --- and thus, the network's connectivity --- by 8.9\% city-wide and understates the average edge length by 6.4\% city-wide.

However, these indicators' values vary across the city. To explore this statistical variation, Table \ref{tab:samples_city} presents summary statistics of these planarity indicators across 100 square-mile samples of Oakland's drivable street network. Our samples' mean $\phi$ and $\lambda$ scores are reasonably close to the city-wide values. However, the samples range from spatial planarity lows of 56.9\% ($\phi$) and 63.7\% ($\lambda$) up to highs of 100\%. 67\% of the samples pass the formal planarity test; however, 63\% of the samples are at least somewhat spatially nonplanar (i.e., with $\phi < 1$).



\section{Discussion}

\subsection{Are street networks planar graphs?}

Our findings suggest that the street networks at the centers of most major cities are formally nonplanar. However, this depends on the scale of measurement: across an entire city there is likely to be at least one overpass or underpass somewhere, while individual neighborhoods or small towns might be formally planar in their entirety. The type and era of urbanization represent another factor. Medieval European towns or informal settlements in the global south may contain fewer grade-separated roads --- and thus are more planar --- than 20th-century American or 21st-century Chinese metropolises. This is a result of the prevailing transportation technologies when the urban form was developed, as well as local terrain, wealth, culture, and politics \citep{southworth_street_1995}.

Street networks are frequently spatially nonplanar because they are embedded in three dimensions, not two: they have a z-coordinate (elevation) along with their x- and y-coordinates. But because they are usually \enquote{mostly} planar (average drivable $\phi$ of 88.5\% and average walkable $\phi$ of 92\%), typically with only a few overpasses or underpasses, they could often be described as \emph{approximately planar}. However, claiming that urban street networks universally are \enquote{planar} misrepresents them in several ways:

\begin{enumerate}
	\item{Intersection counts are overestimated due to false nodes where grade-separated edges cross}
	\item{Average edge lengths are underestimated}
	\item{Connectivity is misrepresented for routing, accessibility analysis, and other topological studies}
\end{enumerate}

Our results demonstrate how this is a bigger problem in Los Angeles than in Florence, but even in Florence the walking network is spatially and formally nonplanar due to the \textit{sottopassaggio} (pedestrian subway) near Stazione di Santa Maria Novella, its central train station. Our results suggest that spatial planarity is inconsistent both across cities as well as across different neighborhoods within individual cities.

The problem with planar models is particularly pronounced around the downtowns of North American cities, due to the prevalence of freeways, bridges, and underpasses. Drivable networks are affected by these in particular. Walkable networks are more affected by pedestrian flyovers and subways, as in Florence. However, even networks of non-freeway, non-pedestrian-only surface streets could easily be nonplanar due to bridges or tunnels in hilly neighborhoods or over rivers.

Contrary to some of the statements in the urban studies literature, our results suggest that it cannot be universally claimed that urban street networks are planar graphs. A graph is not planar because its edges \emph{usually} intersect only at nodes: by definition it is planar because its edges \emph{exclusively} intersect at nodes. However, as \citet{newman_networks:_2010} points out, debating the semantics of formal planarity may be missing the point -- more interesting is the \emph{extent} to which a network is planar.

\subsection{Are street networks well-modeled by planar graphs?}

As George Box famously said, \enquote{All models are wrong but some are useful.} Even if they are not formally planar, can street networks be simplified to planar graphs and still be usefully well-modeled? Our results suggest that the answer depends on the study site and on the type of analysis. In limited circumstances --- where the circulation network exhibits few (or ideally zero) underpasses, overpasses, or grade-separation --- perhaps yes.

But universally, we cannot answer yes. Most egregiously, imposing planarity on a nonplanar street network forces false nodes at underpasses and overpasses, breaking routing and network-based accessibility modeling. For this reason, nonplanar graphs have been the standard for decades in transportation engineering, real-world traffic assignment models, and routing engines.

But planar graphs are often used in the literature to characterize urban form and morphology. So, aside from routing, do planar graphs offer \emph{useful} models for this type of research? Again, only in limited circumstances, such as the drivable networks in the three Italian cities we analyzed. Elsewhere, the results in Table \ref{tab:world_cities} showed how common urban form measures such as intersection counts are overstated by planar models (16\% on average in the drivable networks), while average street segment lengths are consequently understated (10\% on average in the drivable networks). Moreover, this misrepresentation behaves inconsistently from place to place: Figure \ref{fig:world_map_bw} and Table \ref{tab:samples_city} demonstrated how the magnitude of bias varies across cities and modes of urbanization. Although planar graphs offer computationally tractable models and allow for the polygonal spatial analysis of urban blocks, they often model these street networks poorly when compared to nonplanar graphs.


\subsection{What can nonplanarity tell us?}

We might repurpose this discrepancy between planar and nonplanar representations of a street network to reveal something useful about urban form. The $\phi$ and $\lambda$ scores of the major US city centers indicate the greater three-dimensionality of their transportation infrastructure compared to that of the European city centers. These US city centers feature more automobile-oriented roadways and are often bounded by grade-separated freeways. By contrast, these European city centers feature less three-dimensionality as their streets and paths tend to be at-grade. Examining how $\phi$ and $\lambda$ change over time in different cities could reveal the type and pace of urban development. 

Most street networks are formally nonplanar while demonstrating $\phi$ and $\lambda$ values that are only slightly nonplanar. Approximate planarity is common across cities because of costs, technology, and politics: it is expensive to engineer a three-dimensional network with z-axis extents (nearly) as extensive as the x- and y-axes. Furthermore, a city fully consumed by overpasses and tunnels would degrade livability and walkability, making it politically infeasible.

\section{Conclusion}

Urban researchers often use planar graphs to model street networks. This study demonstrated that, although in limited circumstances these models may be accurate, they behave inconsistently across different kinds of cities by misrepresenting connectivity, accessibility, routing, intersection counts and densities, and street segment lengths. In most circumstances, nonplanar graphs provide more faithful models. It also demonstrated how these indicators can characterize urbanization in different cities, particularly through transportation infrastructure's three-dimensionality.

Future research can further explore this latter finding, as it likely correlates with other measures of urbanization, development, and era. Finally, future research might examine how nonplanar intersection counts represent true intersections if multiple adjacent edges form multiple graph intersections at a point where only one true intersection exists from an urban design perspective.


%\clearpage
\bibliographystyle{apalike}
\bibliography{references}

\end{document}
