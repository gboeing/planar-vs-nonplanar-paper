\begin{tabular}{ | p{\textwidth} | }

\hline

\enquote{In a planar graph, no links intersect, except by nodes. This feature represents a transportation network well.} \citep[p.~6]{dill_measuring_2004} \\ \hline

\enquote{Street networks are planar graphs composed of junctions and street segments...} \citep[p.~18]{batty_network_2005} \\ \hline

\enquote{Any of these street networks (SNS) can be described by an embedded planar graph... Street networks are planar graphs and such planarity strongly constrains their heterogeneity...} \citep[pp.~514~\&~521]{buhl_topological_2006} \\ \hline

\enquote{Planar graphs are those graphs forming vertices whenever two edges cross, whereas nonplanar graphs can have edge crossings that do not form vertices. The graphs representing urban street patterns are, by construction, planar...} \citep[p.~3]{cardillo_structural_2006} \\ \hline

\enquote{The connection and arrangement of a road network is usually abstracted in network analysis as a directed planar graph...} \citep[p.~340]{xie_measuring_2007} \\ \hline

\enquote{Urban street patterns form planar networks... The simplest description of the street network consists of a graph whose links represent roads and whose vertices represent road intersections and end points. For these graphs, links intersect essentially only at vertices and are thus planar.} \citep[p.~1]{barthelemy_modeling_2008} \\ \hline

\enquote{Urban street networks as spatial networks are embedded in planar space, which give many constraints.} \citep[p.~1]{hu_topological_2008} \\ \hline

\enquote{...a street network is a strange network when compared to other social or biological networks in the sense that it is embedded in the Euclidian [sic] space and the edges do not cross each other. In graph theory, such a network is called a planar graph.} \citep[p.~259]{masucci_random_2009} \\ \hline

\enquote{...street networks are embedded in space and are planar in nature...} \citep[p.~114]{porta_networks_2010} \\ \hline

\enquote{Roads, rail, and other transportation networks are spatial and to a good accuracy planar networks. For many applications, planar spatial networks are the most important...} \citep[p.~3]{barthelemy_spatial_2011} \\ \hline

\enquote{...urban road systems can be (in good approximation) considered as planar networks, i.e., links cannot \enquote{cross} each other without forming a physical intersection (node) as long as there are no tunnels or bridges... The meaningful definition of link angles requires the presence of a planar network, which is assumed to be the case in urban road systems.} \citep[pp.~563~\&~567]{chan_urban_2011} \\ \hline

\enquote{Road networks are planar graphs consisting of a series of land cells surrounded by street segments.} \citep[p.~3]{strano_elementary_2012} \\ \hline

\enquote{Planar graphs are basic tools for understanding transportation systems embedded in two-dimensional space, in particular urban street networks... As these graphs are embedded in a two-dimensional surface, the
planarity criteria requires that the links do not cross each other.} \citep[p.~1]{masucci_limited_2013} \\ \hline

\enquote{...street networks are essentially planar; in the absence of tunnels and bridges, the streets (the links) cannot cross without generating an intersection or a junction, that is, a node.} \citep[p.~1]{gudmundsson_entropy_2013}. \\ \hline

\enquote{Networks of street patterns belong to a particular class of graphs called planar graphs, that is, graphs whose links cross only at nodes.} \citep[p.~1074]{strano_urban_2013} \\ \hline

\enquote{In city science, planar networks are extensively used to represent, to a good approximation, various infrastructure networks... in particular, transportation networks and more recently streets patterns...} \citep[p.~1]{viana_simplicity_2013} \\ \hline

\enquote{...finding a typology of street patterns essentially amounts to classifying planar graphs...} \citep[p.~2]{louf_typology_2014} \\ \hline

\enquote{...we are dealing with spatial graphs, which tend to be planar...} \citep[p.~2191]{zhong_detecting_2014} \\ \hline

\enquote{Urban transport systems as networks can be represented as planar graphs...} \citep[p.~2]{wang_resilience_2015} \\ \hline

\enquote{Modeling a road network as a planar graph seems very natural.} \citep[p.~42]{aldous_routed_2016} \\ \hline

\enquote{In city science, planar networks are extensively used to represent various infrastructure networks. In particular, transportation networks and street patterns...} \citep[p.~257]{barthelemy_paths_2017} \\ \hline

\enquote{In graph theory, a spatial street network is a type of planar graph embedded in Euclidean space.} \citep[p.~168]{law_defining_2017} \\ \hline

\end{tabular}
